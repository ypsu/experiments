\documentclass[10pt,a4paper,twocolumn]{article}

\usepackage{verbatim}
\usepackage[utf8x]{inputenc}
\usepackage{textcomp}
\usepackage{amsmath}
\usepackage{amssymb}
\usepackage{amsthm}
\usepackage{latexsym}
\usepackage[english]{babel}
\usepackage{listings}
\usepackage[left=0.15cm,top=0.15cm,bottom=0.30cm,right=0.15cm,nohead,nofoot]{geometry}
\sloppy
\parindent=0pt
\lstset{
	basicstyle=\small\ttfamily,
	language=c++,
	tabsize=4,
}

\usepackage[pdftex,colorlinks,citecolor=black,filecolor=black,linkcolor=black,
urlcolor=black]{hyperref}

\newcommand{\Cov}{\text{Cov}}
\newenvironment{fpar}
	{
		\nopagebreak
		\begin{tabular}{|l|}
		\hline\begin{minipage}{277pt}
		\vspace{0.5em}
	}
	{
		\vspace{0.5em}
		\end{minipage}\\
		\hline\end{tabular}
		\nopagebreak
	}

\begin{document}

\begin{fpar}
{\bf Vector products}

$\alpha$ - the angle between $a, b$.

$\cos \alpha = $ dot product of normalized vectors\\
dot product = $(a_x\cdot b_x + a_y\cdot b_y) / (|a|\cdot |b|)$.\\
cross product = $a_x\cdot b_y - a_y\cdot b_x$.
\end{fpar}

\begin{fpar}
{\bf Vector rotation}
$$x' = x\cos\alpha - y\sin\alpha$$
$$y' = x\sin\alpha + y\cos\alpha$$
\end{fpar}

\begin{fpar}
{\bf Triangle area}

\emph{Heron's formula}:
$$s = \frac{a+b+c}2$$
$$A = \sqrt{s(s-a)(s-b)(s-c)}$$

\emph{Heron's formula - stable}:
$$a \geq b \geq c$$
$$A = \frac14 \sqrt{(a+(b+c))(c-(a-b))(c+(a-b))(a+(b-c))}$$

\emph{Cross product}:
Two vectors starting from [0,0]:
$$A = \frac12 |x_1y_2 - x_2y_1|$$
\end{fpar}

\begin{fpar}
{\bf Simple polygon's area}

Let $x_n,y_n = x_0,y_0$. Then:
$$A = \frac12 \sum_{i=0}^{n-1} (x_i y_{i+1} - x_{i+1}y_i)$$
The vertices must be ordered counterclockwise. If they are ordered clockwise
the area will be negative but correct in absolute value.
\end{fpar}

\begin{fpar}
{\bf Pick's theorem}

If a polygon's vertices are grid points:

$$A = i + \frac 12b - 1$$
$A$ - area of the polygon\\
$i$ - inner points\\
$b$ - points on the boundary\\
\end{fpar}

\begin{fpar}
{\bf Point-line distance}

Point: $(x_0, y_0)$\\
Line: $[(x_1, y_1), (x_2, y_2)]$\\

$$d = \frac{\left|(x_2-x_1)(y_1-y_0) - (x_1-x_0)(y_2-y_1)\right|}
{\sqrt{(x_2-x_1)^2 + (y_2-y_1)^2}}$$
\end{fpar}

\begin{fpar}
{\bf Line-line intersection}

Line 1: $[(x_1,y_1), (x_2,y_2)]$\\
Line 2: $[(x_3,y_3), (x_4,y_4)]$

$$x = \frac{(x_1 y_2-y_1 x_2)(x_3-x_4)-(x_1-x_2)(x_3 y_4-y_3
x_4)}{(x_1-x_2)(y_3-y_4)-(y_1-y_2)(x_3-x_4)}
$$
$$y = \\ \frac{(x_1 y_2-y_1
x_2)(y_3-y_4)-(y_1-y_2)(x_3 y_4-y_3
x_4)}{(x_1-x_2)(y_3-y_4)-(y_1-y_2)(x_3-x_4)}
$$
\end{fpar}

\begin{fpar}
{\bf Line-circle intersection}

Line: $[(x_1,y_1), (x_2,y_2)]$\\
Circle: $[(0,0), r]$
$$d_x = x_2 - x_1\qquad d_y=y_2-y_1\qquad
d_r=\sqrt{d_x^2 + d_y^2}$$
$$D=x_1y_2 - x_2y_1\qquad
\Delta = r^2 d_r^2 - D^2$$
$\Delta < 0 \implies \text{no intersection}$\\
$\Delta = 0 \implies \text{tangent}$\\
$\Delta > 0 \implies \text{intersection}$\\
$$x = \frac{Dd_y \pm \text{sgn}^*(d_y)d_x\sqrt\Delta }{d_r^2}$$
$$y = \frac{-Dd_x \pm |d_y|\sqrt\Delta }{d_r^2}$$
where
$$\text{sgn}^*(x) =
\left\{\begin{array}{ll}
-1 & \text{for } x < 0 \\
1 & \text{otherwise}
\end{array}
\right.
$$
\end{fpar}

\begin{fpar}
{\bf Circle-circle intersection}

$$(x-x_0)^2 + (y-y_0)^2 = r_0^2$$
$$x^2 + y^2 = r_1^2$$
\\
$$d = x_0^2 + y_0^2$$
$$p=\sqrt{\left((r_0+r_1)^2 - d\right)
          \left(d-(r_1-r_0)^2\right)}$$
\\
$$A_x = \frac{x_0}2 + \frac{y_0p - x_0\left(r_0^2-r_1^2\right)}{2d}$$
$$A_y = \frac{y_0}2 + \frac{-x_0p - y_0\left(r_0^2-r_1^2\right)}{2d}$$
\\
$$B_x = \frac{x_0}2 + \frac{-y_0p - x_0\left(r_0^2-r_1^2\right)}{2d}$$
$$B_y = \frac{y_0}2 + \frac{x_0p - y_0\left(r_0^2-r_1^2\right)}{2d}$$
\end{fpar}

\begin{fpar}
{\bf Quadratic equation}

$$ax^2 + bx + c = 0, \quad b \neq 0$$
$$t = -\left(b + \text{sgn}(b) \sqrt{b^2-4ac}\right) / 2$$
$$x_1 = t / a$$
$$x_2 = c / t$$
\end{fpar}

\begin{fpar}
{\bf Pythagorean Triplets}

$$a^2 + b^2 = c^2$$
\emph{Euclid's generating formula}:\\
$a = v^2 - u^2$\\
$b = 2uv$\\
$c = u^2 + v^2$

where $u$ and $v > u$ are relatively prime and of opposite parity.
The formula generates a set of distinct triples containing precisely the
primitive triples (after appropriately sorting ($v^2-u^2$) and $2uv$).
\end{fpar}

\begin{fpar}
{\bf Primes}

32003\\
32009\\
65003\\
65011\\

999 999 929\\
999 999 937\\
1 000 000 007\\
1 000 000 009\\

2 147 483 549\\
2 147 483 563\\

999 999 999 999 999 967\\
999 999 999 999 999 989\\
1 000 000 000 000 000 003\\
1 000 000 000 000 000 009\\

9 223 372 036 854 775 643\\
9 223 372 036 854 775 783
\end{fpar}

\begin{fpar}
{\bf Divisors}

Let $n = p^a$, where $p$ is a prime.\\
Number of divisors: $(a+1)$.\\
Sum of divisors: $(p^{a+1} - 1) / (p-1)$.\\
Both are multiplicative functions.
\end{fpar}

\begin{fpar}
{\bf Euler's totient function}

$\varphi(n)$ the number of positive integers less than or equal to $n$ that are
coprime to $n$.
$$\sum_{d|n} \varphi(d) = n$$
$$\varphi(p^k) = (p-1) p^{k-1}$$
Examples: $\varphi(9) = 6$, $\varphi(36) = 12$\\
It is multiplicative.
$n$ and $a$ coprime $\implies a^{\varphi(n)}\equiv 1 \mod n$.
\end{fpar}

\begin{fpar}
{\bf Sums}

$$\sum_{k=1}^n k = \frac{n (n+1)}2$$
$$\sum_{k=1}^n k^2 = \frac{n (n+1) (2n+1)}6$$
$$\sum_{k=1}^n k^3 = \left(\frac{n (n+1)}2\right)^2$$
$$\sum_{k=0}^n x^k = \frac{x^{n+1} - 1}{x - 1}$$
\end{fpar}

\begin{fpar}
{\bf Probability}

Conditional probability:
$$P(A|B) = \frac{P(A \cap B)}{P(B)}$$

Bayes' theorem:
$$P(A|B) = \frac{P(A) P(B|A)}{P(B)}$$
\end{fpar}

\begin{fpar}
{\bf Euler's formula for planar graphs}
$$vertices - edges + faces = 2$$
\end{fpar}

\begin{fpar}
{\bf String hashing}

\begin{lstlisting}
unsigned long djb2(unsigned char *str)
{
	unsigned long hash = 5381;
	int c;
	while (c = *str++)
		hash = hash*33 + c;
	return hash;
}
\end{lstlisting}
\end{fpar}

\begin{fpar}
{\bf Gray code}

Gray code is a binary numeral system where two successive values differ in only
one bit.\\
\verb+G(n) = n ^ (n >> 1)+
\end{fpar}

\begin{fpar}
{\bf Lexicographically next bit permutation}

Example: N = 3: $\ldots$, 00010101, 00010110, 00011001, $\ldots$.

\begin{lstlisting}
unsigned int v; // current permutation of bits 
unsigned int w; // next permutation of bits

unsigned int t = v | (v - 1);
w = (t + 1) | (((~t & -~t) - 1)
	>> (__builtin_ctz(v) + 1));  
\end{lstlisting}
\end{fpar}

\begin{fpar}
{\bf GCC hacks}

You can append l or ll for long and long long variants.\\
\\
\verb+int __builtin_ffs(unsigned int x)+\\
Returns one plus the index of the least significant 1-bit of x, or if x is
zero, returns zero.\\
\\
\verb+int __builtin_clz(unsigned int x)+\\
Returns the number of leading 0-bits in x, starting at the most significant bit
position. If x is 0, the result is undefined.\\
\\
\verb+int __builtin_ctz(unsigned int x)+\\
Returns the number of trailing 0-bits in x, starting at the least significant
bit position. If x is 0, the result is undefined.\\
\\
\verb+int __builtin_popcount(unsigned int x)+\\
Returns the number of 1-bits in x.\\
\\
\verb+T std::__gcd(T a, T b)+\\
Returns the greatest common divisor of two integer values.\\
\\
\verb+T __gnu_cxx::power(T x, int n)+\\
Returns $x^n$, where $n >= 0$. Include file: \verb+ext/numeric+.
\end{fpar}

\end{document}
